\documentclass[12pt]{article}
\usepackage[a4paper, tmargin=0.75in, lmargin=0.80in, rmargin=0.80in, bmargin=1in]{geometry}
\usepackage{hyperref}
%\usepackage{multicol}
\hypersetup{
    colorlinks=true,
    linkcolor=black,
    filecolor=magenta,      
    urlcolor=blue,
    citecolor=black,
}
\usepackage[utf8]{inputenc}
\usepackage[T1]{fontenc}
\usepackage[english]{babel}

\usepackage{amsmath}
\usepackage{amsfonts}
\usepackage{amssymb}
\usepackage{bm}
\usepackage{siunitx}
\sisetup{per-mode = symbol}
\usepackage[document]{ragged2e}
\usepackage{setspace}
\usepackage{parskip}


\begin{document}
% Global spacing settings
\setstretch{1.2}
\setlength{\parskip}{10pt}
\setlength{\parindent}{0pt}

\title{Error models for the water surface elevation and discharge simulations}
\author{Felipe Méndez-Ríos\textsuperscript{1}, Benjamin Renard\textsuperscript{2} \\ 
\textsuperscript{1}PhD student at INRAE Lyon, France \\
\textsuperscript{2}Researcher at INRAE Aix-en-Provence,France \\
} 
\date{24 july 2025}
\maketitle

\section{\textbf{Model and hypotheses}}

\subsection{\textbf{Measurements errors:}}

\subsubsection{\textbf{Gauged WSE errors:}}

We have \(N\) measurements of the WSE, (\(\tilde{H}_{x_i,t_i})_{i=1,\ldots,N}\) at given positions \(x_i\)  and times \(t_i\). The error model is presented below:

\begin{equation}
    \tilde{H}_{x_i,t_i} = H_{x_i,t_i} + \delta_i^{(H)} \text{,} \quad \text{with} \quad    \delta_i^{(H)} \sim \mathcal{N}\left(0, u_i^{(H)}\right)
\end{equation}

With \( H_{x_i,t_i} \): the true values of the WSE [\si{\meter}];
\(\delta_i^{(H)}\): measurement error [\si{\meter}];
\(u_i^{(H)}\): standard deviation of the error [\si{\meter}]

\subsubsection{\textbf{Gauged discharge errors:}}

We have \(N\) measurements of discharge, (\(\tilde{Q}_{x_i,t_i})_{i=1,\ldots,N}\) at given positions \(x_i\)  and times \(t_i\). The error model is presented below:

\begin{equation}
    \tilde{Q}_{x_i,t_i} = Q_{x_i,t_i} + \delta_i^{(Q)} \text{,} \quad \text{with} \quad    \delta_i^{(Q)} \sim \mathcal{N}\left(0, u_i^{(Q)}\right)
\end{equation}

With \( Q_{x_i,t_i} \): the true values of the discharge [\si{\cubic\meter\per\second}];
\(\delta_i^{(Q)}\): measurement error [\si{\cubic\meter\per\second}];
\(u_i^{(H)}\): standard deviation of the error [\si{\cubic\meter\per\second}].

\subsection{\textbf{WSE and discharge simulations: }}

WSE and discharge simulations are formalized as a function \(\mathcal{M}\) representing the hydrodynamic model, for instance, MAGE code.

\begin{equation}
    (\hat{H}_{x,t},\hat{Q}_{x,t})(\boldsymbol{\theta}) = \mathcal{M}(K(x;\boldsymbol{\theta}),t) \text{,} \quad \text{with} \quad K(x;\boldsymbol{\theta}) = \theta_0 + \sum_{d=1}^{D} \theta_d \, P_d(x)
\end{equation}

With \(\hat{Q}_{x,t}\): discharge [\si{\cubic\meter\per\second}] at the curvilinear coordinate \(x\) along the river and at time \(t\); \(\hat{H}_{x,t}\): water surface elevation [\si{\meter}] at the same position and time; 
\(K\): friction coefficient [\si{\meter^{1/3}\per\second}];     
\(P_d\): the Legendre polynomial of degree \(d\); 
\(\boldsymbol{\theta}\): coefficients related to Legendre polynomials.

Note that the covariate \(x\) can be replaced by another covariate that depends on \(x\), \(C(x)\).


\subsection{\textbf{Structural error:}}

Error models are developed using the water surface elevations (WSE), but they are transferable to discharge by replacing the superscript \((H)\) with \((Q)\).

\begin{equation}
    H_{x,t} = \hat{H}_{x,t} + \varepsilon^{(H)}_{x,t} \text{,} \quad \text{with} \quad \varepsilon^{(H)}_{x,t} \sim \mathcal{N}(0, \sigma^{(H)}_{x,t})
\end{equation}


\begin{equation}
    \sigma^{(H)}_{x,t} = f^{(H)}(x,t,\boldsymbol{\gamma}^{(H)})
\end{equation}

\vspace{0.2cm}

Here some examples of the error model \(f^{(H)}\):

\subsubsection{\textbf{Constant structural standard deviation in space and time:}}

\begin{equation}
    f(x,t,\gamma) = \gamma
\end{equation}

\subsubsection{\textbf{Constant structural standard deviation in time for a given position \(x_0\):}}

\begin{equation}
    f(x_0,t,\gamma_{x_0}) = \gamma_{x_0}
\end{equation}

This model is valid only at position \(x_0\). It is suitable for processing a time series at a given position \(x_0\).
We can repeat this error model at several positions \((x_1,\ldots,x_M)\), but we cannot interpolate it outside these positions. For example, for two positions: 

\begin{equation}
    f(x,t,\boldsymbol{\gamma}) = 
    \begin{cases}
        \gamma_1 \quad \text{if} \quad x=x_1 \\
        \gamma_2 \quad \text{if} \quad x=x_2\\
        ?? \quad \text{if} \quad x \neq x_1 \text{ \& } x \neq x_2
    \end{cases}
\end{equation}

\subsubsection{\textbf{Constant structural standard deviation in space for a given time \(t_0\):}}

\begin{equation}
    f(x,t_0,\gamma_{t_0}) = \gamma_{t_0}
\end{equation}

This model is valid only at time \(t_0\). It is suitable for processing a water surface elevation at a given time \(t_0\). 
We can repeat this error model at several times \((t_1,\ldots,t_M)\), but we cannot interpolate it outside these times.

\subsubsection{\textbf{Structural standard deviation varying in space and time:}}

Here a proposition:

\begin{equation}
    f(x,t,\boldsymbol{\gamma}) = \gamma_1 + \gamma_2 \times \hat{H}_{x,t}
\end{equation}

\subsection{\textbf{Total error:}}

We assume that the structural error is independent of the measurement error of the water surface elevation. By combining the previous equations, we arrive at the following total error model:

\begin{equation}
    \tilde{H}_{x_i,t_i} = \hat{H}_{x_i,t_i}(\boldsymbol{\theta})  + \varepsilon^{(H)}_{x_i,t_i} + \delta_i^{(H)} \quad \text{with} \quad \varepsilon^{(H)}_{x_i,t_i} + \delta_i^{(H)} \sim \mathcal{N}\left(0, \sqrt{\sigma^{2^{(H)}}_{x_i,t_i} + u_i^{2^{(H)}}}\right)
\label{equation:total_error_model}
\end{equation}
 
\section{\textbf{Bayesian estimation}}

\subsection{\textbf{Information brought by the mesaurements: likelihood}}

According to the equation \ref{equation:total_error_model}, the gauged WSE (\(\tilde{H}_{x_i,t_i}\)) is a realisation from a Gaussian distribution with mean \(\hat{H}_{x_i,t_i}(\boldsymbol{\theta})\) (i.e. the simulated water surface elevation) and standard deviation \(\sqrt{\sigma^{2^{(H)}}_{x_i,t_i} + u_i^{2^{(H)}}}\). Assuming that errors affecting all gauged WSE are mutually independent, the likelihood can be written as:

\begin{equation}
    p(\boldsymbol{\tilde{H}} | \boldsymbol{\theta},\boldsymbol{\gamma}) = \prod_{i=1}^{N} d_\mathcal{N}\left(\tilde{H}_{x_i,t_i} ; \hat{H}_{x_i,t_i}(\boldsymbol{\theta}), \sqrt{\sigma^{2^{(H)}}_{x_i,t_i} + u_i^{2^{(H)}}}\right)
\end{equation}

Where \(\boldsymbol{\tilde{H}} = (\tilde{H}_1,\ldots,\tilde{H}_N)\) denote the \(N\) gauged WSE and \(d_\mathcal{N}(z;m,s)\), denote the probability density function (pdf) of a Gaussian distribution with mean \(m\) and standard deviation \(s\) evaluated at some value \(z\).


TO DO:
-Total likelihood for WSE and discharge.
-Prior distribution 
-Posterior distribution

\end{document}